\chapter{Pizza, Quiche  \& Co}

\section{Pizza-Teig}\label{sec:Pizza-Teig}\index{Pizza-Teig}
\subsection*{Zeitplan}
\begin{tabular}{ r @{ Uhr \phantom{bla} } l}
    \toprule
    \multicolumn{1}{c @{\phantom{ bla }}}{Zeit} & \multicolumn{1}{@{}l}{Aktion}           \\ \midrule
    00:00                                       & \Gls{Hauptteig}                         \\ \midrule
    \multicolumn{2}{c @{\phantom{ bla }}}{Schnelle Variante}                \\
    00:20                                       & \Gls{Stockgare}                         \\
    02:00                                       & \Gls{Formen}                            \\
    02:10                                       & \Gls{Stueckgare}                        \\
    03:10                                       & \Gls{Backen}                            \\ \midrule
    \multicolumn{2}{c @{\phantom{ bla }}}{Kalte Variante}                 \\
    00:20                                       & \Gls{Stockgare} 1                       \\
    00:50                                       & \Gls{DehnenUndFalten} + \Gls{Stockgare} 2 \\
    05:00 / 09:00                               & \Gls{Formen}                            \\
    05:10 / 09:10                               & \Gls{Stueckgare}                        \\
    06:10 / 10:10                               & \Gls{Backen}                            \\ \bottomrule
\end{tabular}

\subsection*{Zutaten für 6 Pizzen}
\begin{tabular}{r l}
    10\;g & frische Hefe  \\
    600\;ml & kaltes Wasser \\
    1\;kg & Mehl          \\
    8\;g & Salz
\end{tabular} 


\subsection*{Zubereitung}

Es gibt die schnelle und die \glqq langsame\grqq\ Variante
\begin{enumerate}
    \item [\Gls{Hauptteig}] Hefe in 600 ml kaltes Wasser bröseln und 5 Minuten gut durch rühren.\\
    Mehl mit Salz zum Hefewasser geben und zu einem Teigkloß verarbeiten. Teig 10 Minuten mit langsam und dann ca. 5 Minuten schnellermit knetenden Händen (Maschine funktioniert nicht wirklich) auf einer leicht bemehlten Oberfläche 10-15 Minuten geschmeidig kneten.
    \item  \textbf{Langsame Variante}
    \begin{itemize}
        \item Teig in eine große geölte Plastikdose geben und nach 30 Minuten dehnen und falten. Danach bis zu 8 Stunden in den Kühlschrank stellen. 
        \item 2 Stunden vor dem Backen den Teig auf die Arbeitsfläche gleiten lassen, falten und dehnen, wieder in die Dose geben.
        \item Abgedeckt nochmals 30 Minuten bei Zimmertemperatur gehen lassen. 
    \end{itemize}
    \item \textbf{Schnelle Variante}
    \begin{itemize}
        \item Teig Schüssel geben und luftdicht abschließen.
        \item Abgedeckt 1:30 Minuten bei Zimmertemperatur gehen lassen. 
    \end{itemize}
    \item [\Gls{Formen}] Teig auf der leicht bemehlten Arbeitsfläche zu einer Rolle formen, dabei nicht zu viel kneten.
    Rolle in 6 gleich große , ca. 250 g schwere Stücke teilen.
    Teigstücke zu Kugeln formen, mit ca. 10 cm Abstand in eine leicht bemehlte Form legen. 
    \item [\Gls{Stueckgare}] Mit Mehl bestäuben und abgedeckt bei Zimmertemperatur eine Stunde gehen lassen.
    \item [\Gls{Backen}] Teigkugeln jeweils mit den Händen auf der bemehlten Arbeitsfläche von innen nach außen zu dünnen runden oder ovalen Fladen drücken, den dabei entstehenden Rand nicht flachdrücken. \\
    Fladen jeweils auf ein Stück Backpapier geben, belegen und backen.
\end{enumerate}
%
\section[Pizzateig mit wenig Hefe und LM]{Pizzateig mit wenig Hefe und Lievito Madre \textmd{(siehe \cite{sonjaPizza}})}\label{sec:Pizza-Teig-2}\index{Pizza-Teig}
\subsection*{Zeitplan}
\begin{tabular}{ r @{ Uhr \phantom{bla} } l}
    \toprule
    \multicolumn{1}{c @{\phantom{ bla }}}{Zeit} & \multicolumn{1}{@{}l}{Aktion}           \\ \midrule
    00:00                                       & \Gls{Autolyse}                         \\ 
    00:20                                       & \Gls{Hauptteig}                         \\ 
    00:40                                       & \Gls{Stockgare}                         \\
    01:40                                       & \Gls{Formen}                            \\
    01:50                                      & \Gls{Stueckgare}                        \\
    13:50  / 73:50                                     & \Gls{Akklimatisieren}                            \\ 
    15:00  / 75:00                                       &  \Gls{Backen}                            \\ \bottomrule
\end{tabular}

\subsection*{Zutaten für 4 Pizzen}
\begin{tabular}{r l}
    60\;g & Lievito Madre aufgefrischt  \\
    2\;g & frische Hefe  \\
    380\;ml & kaltes Wasser \\
    600\;g & Tipo 00          \\
    12\;g & Salz
\end{tabular} 


\subsection*{Zubereitung}

Es gibt die schnelle und die \glqq langsame\grqq\ Variante
\begin{enumerate}
    \item [\GLS{Autolyse}] Wasser und Mehl kurz vermischen. Abgedeckt für 20 Minuten quellen lassen.
    
    \item [\Gls{Hauptteig}] Lievito Madre und Hefe hinzufügen, für 8−10 Minuten langsam kneten. Salz hinzugeben und für weitere 4−8 Minuten schnell auskneten.
    \item [\Gls{Stockgare}] Den Teig in eine geölte Teigwanne geben und für 60 Minuten bei Raumtemperatur (22−22 °C) reifen lassen.
    Dabei nach 30 Minuten einmal dehnen und falten.
    \item [\Gls{Formen}]  Anschließend den Teig in 4 gleich große Teiglinge teilen. Jeweils zu einer 4 Kugeln rund wirken.
    \item [\Gls{Stueckgare}]  Die Kugeln leicht einölen und zurück in die Teigwanne setzen. Für insgesamt 12−72 Stunden in einer leicht geölten Schüssel oder Teigwanne in den Kühlschrank (4−5 °C) geben.
    \item [\GLS{Akklimatisieren}] Rechtzeitig den Backofen so heiß wie möglich vorheizen, möglichst auf 280−300 °C Ober-/Unterhitze,\\
    Den Teig nach der kalten Gare mindestens 30−60 Minuten akklimatisieren lassen.
    \item[Belegen] 
    Die Teigballen aus der Teigwanne entnehmen und auf die gut mit Hartweizenmehl bemehlte Arbeitsfläche geben.\\
    Vorsichtig ringsherum mit den Fingerspitzen einen Rand abdrücken.
    Anschließend behutsam von innen nach außen etwas auseinanderziehen und -dehnen. (Nicht ausrollen!!!)\\
    Die Pizzaiola (sparsam) auf dem Teig verteilen und mit Mozzarella belegen.\\
    Danach mit den weiteren Zutaten nach Wahl belegen.\\
    Nach Belieben zusätzlich mit etwas Olivenöl beträufeln und eventuell mit getrocknetem Oregano bestreuen. (Frische Kräuter erst nach dem Backen auf die Pizza geben.)
    \item [\Gls{Backen}] Bei 280−300 °C jeweils für ca. 6−10 Minuten backen. 
\end{enumerate}
%

\section{Pizzaiola}
\subsection*{Zutaten für 4 Pizzen}

\begin{tabular}{r l}
         400\;g & Tomaten aus der Dose                                     \\
          30\;g & getrocknete Tomaten in Öl eingelegte getrocknete Tomaten \\
          10\;g & Olivenöl                                                 \\
          1\;TL & getrockneter Oregano                                     \\
            1\; & Knoblauchzehe optional                                   \\
    nach Bedarf & Salz
\end{tabular} 
\subsection*{Zubereitung}
Alle Zutaten für die Pizzaiola je nach Belieben grob bis fein pürieren.


\section{Quiche} \label{sec:Quiche}\index{Quiche}
\subsection*{Allgemeines}
\begin{tabular}{lrl}
    Personen         &   4 &  \\
    Zubereitungszeit &   40 & Minuten \\
    Woher & & ?%\cite[vgl.][]{OttoLenghiJerusalem}
\end{tabular} 

\subsection*{Zutaten}
\begin{tabular}{lrl}
    \multicolumn{3}{l}{\textbf{Teig}}                      \\
    250 &          g & Mehl                                \\
    150 &         ml & Butter                              \\
    1   &            & Eigelb                              \\
    4   &         EL & kaltes Wasser                       \\
    &            & Salz                                \\
    \multicolumn{3}{l}{\textbf{Dinkel Teig}}               \\
    210 &          g & Dinkelmehl 630                      \\
    50  &          g & Weizen 405                          \\
    125 &         ml & Butter                              \\
    1   &            & Ei                                  \\
    3   &         EL & kaltes Wasser                       \\
    30  & G Parmesan &                                     \\
    &            & Salz                                \\
    \multicolumn{3}{l}{\textbf{Füllung}}                   \\
    300 &          g & Käse (käftig)                       \\
    150 &          g & Schinkespeck (ggf. ersetzen)        \\
    1   &            & Zwiebel                             \\
    4   &          g & Eier + 1 Eiweiß                     \\
    200 &          g & Schmand oder saure Sahne            \\
    1   &       Bund & Schnittlauch                        \\
    &            & Salz, Pfeffer, 1 TL Paprika edelsüß
\end{tabular} 
\subsection*{Zubereitung}
\begin{enumerate}
    \item Alle Zutaten für den Teig bei niedriger Stufe mit Knethaken gut vermengen. 
    \item Den fertigen Teig mindestens 30 Minuten in den Kühlschrank stellen.
    \item Eine Quiche Form einfetten. Den Teig in die Form geben, mit einer Gabel einstechen und bei 200 Grad ca. 20 Minuten backen.
    \item Die Zutaten für die Füllung vermengen und auf den Teig geben und bei 200 Grad 30 Minuten garen. In der Mitte mit einer Gabel prüfen, ob die Füllung gestockt hat. Wenn nicht weiter backen, dann aber abdecken.
\end{enumerate}

\section{Crespelle} \label{sec:Crespelle}\index{Crespelle}
\subsection*{Allgemeines}
%\begin{tabular}{lrl}
%    Personen         &   4 &  \\
%    Zubereitungszeit &   40 & Minuten \\
%    Woher & & ?%\cite[vgl.][]{OttoLenghiJerusalem}
%\end{tabular} 

\subsection*{Zutaten}
\begin{tabular}{lrl}
    120 &          g & Tipo 00                      \\
    250 &         ml & Milch                             \\
    2   &            & Eier                                  \\
    &            & Salz, Pfeffer                                \\
\end{tabular} 
%\subsection*{Zubereitung}
%\begin{enumerate}
%    \item Alle Zutaten für den Teig bei niedriger Stufe mit Knethaken gut vermengen. 
%    \item Den fertigen Teig mindestens 30 Minuten in den Kühlschrank stellen.
%    \item Eine Quiche Form einfetten. Den Teig in die Form geben, mit einer Gabel einstechen und bei 200 Grad ca. 20 Minuten backen.
%    \item Die Zutaten für die Füllung vermengen und auf den Teig geben und bei 200 Grad 30 Minuten garen. In der Mitte mit einer Gabel prüfen, ob die Füllung gestockt hat. Wenn nicht weiter backen, dann aber abdecken.
%\end{enumerate}
