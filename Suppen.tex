\chapter{Suppen}

\section{Grießklöße}\index{Rindfleischsuppe Grießklöße}\index{Grießklöße}
\subsection*{Allgemeines}
\begin{tabular}{lrl}
	Personen         &                        4 &  \\
	Zubereitungszeit &                       15 & Minuten \\
\end{tabular} 

\subsection*{Zutaten}
\begin{tabular}{rll}
	0,25 & l     & Milch           \\
	   1 & Prise & Salz            \\
	  20 & g     & Butter          \\
	 100 & g     & Hartweizengries \\
	   1 &       & Ei              \\
	   2 & EL    & Kräuter
\end{tabular} 
\subsection*{Zubereitung}
\begin{enumerate}
	\item Milch mit Salz und Butter zum Kochen bringen.
	\item Platte auf 1 stellen (Induktion ) und Grieß, Ei und Kräuter langsam unterrühren.
	\item Masse noch 1 Minute weiter rühren und dann erkalten lassen (Nächster Tag geht auch).
	\item Klöße formen (feuchte Hände) und ab in die heiße Suppe. Ziehen lassen, nicht kochen. Servieren, wenn die Klöße aufsteigen.
\end{enumerate}

\section{Orangen Tomatensuppe}\index{Tomatensuppe}\index{Orangen}\index{Tomaten!Suppe}
\subsection*{Allgemeines}
\begin{tabular}{lrl}
	Personen         &                        4 &  \\
	Zubereitungszeit &                       20 & Minuten \\
\end{tabular} 

\subsection*{Zutaten}
\begin{tabular}{rll}
	 0,25 & l      & Orangensaft                                               \\
	    1 & Pck.   & pürierte Tomaten, oder entsprechende Menge Fleischtomaten \\
	  1/4 & g Tube & Tomatenmark                                               \\
	  100 & g      & Frischkäse                                                \\
	  100 & ml     & Sahne                                                     \\
	    2 & EL     & Rinderbrühe                                               \\
	  1/4 & l      & Wasser                                                    \\
	    1 & EL     & brauner Zucker                                            \\
	    1 &        & Chillieschote                                             \\
	    1 & TL     & Kurkuma                                                   \\
	etwas &        & kleingeschittene Paprika
\end{tabular} 
\subsection*{Zubereitung}
\begin{enumerate}
	\item Bis auf den Doppelrahmfrischkäse alle Zutaten in einen Topf geben und zusammen 5 Minuten kochen lassen. Dann den Doppelrahmfrischkäse zugeben und mit dem Pürierstab kurz aufschlagen. Mit Salz und Pfeffer abschmecken. 
	\item Auf flachen Tellern portionieren und mit gestoßenen rosa Beeren und klein geschnittenen roten Paprikawürfeln garniert servieren.
\end{enumerate}

\todo[inline]{Kolrabi Creme }
\todo[inline]{Rindfleisch}
\todo[inline]{Hühner }