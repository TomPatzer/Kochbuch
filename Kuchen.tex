\chapter{Kuchen}
\section{Rockmannkuchen}
\subsection*{Zubereitung}
\begin{tabular}{lrr}
	Personen         &                12 &  \\
	Zubereitungszeit &               40 & Minuten \\
	Gesamtzeit       &                   & Stunden \\
%	Entnommen aus    & entnommen aus \cite{Oliver2009} &
\end{tabular} 

\subsection*{Zutaten}
\begin{tabular}{lrr}
	Preiselbeeren        &   1 &   Glas \\
	Blockschokolade      & 100 &      g \\
	Zucker               & 100 &      g \\
	gemahlene Haselnüsse & 100 &      g \\
	Backpulver           &   2 &     TL \\
	Wasser               &   3 &     TL \\
	Rum                  &   3 &   TL g \\
	Sahne                &   2 & Becher \\
	Eier                 &   4 &
\end{tabular} 

\subsection*{Zubereitung}
\begin{enumerate}
	\item Eigelb und Zucker schaumig schlagen
	\item Nüsse, Schokolade, Backpulver und Wasser zugeben. 
	\item Eischnee darunterheben
	\item 30 min. bei 170 Grad backen
	\item Tortenboden mit 3 EL Rum beträufeln, Preisselbeeren darauf.
	\item Sahne mit Sahnesteif geschlagen darüber
\end{enumerate}
  
\section{Zitronentarte mit Himbeersoße} \index{Zitronentarte}\index{Zitronen!Tarte}\index{Tarte}\index{Zitrone}
\subsection*{Zubereitung}
\begin{tabular}{lrl}
	Personen         &                12 &  \\
	Zubereitungszeit &               100 & Minuten \\
	Gesamtzeit       &                   & Stunden \\
	Entnommen aus    & entnommen aus \cite{Oliver2009} &
\end{tabular} 

\subsection*{Zutaten}
\begin{tabular}{lrl}
	Himbeeren     & 200 &     g \\
	Bio-Zitronen  &   5 & Stück \\
	Zucker        &     &  \\
	Puderzucker   & 250 &     g \\
	Vanilleschote &   1 &  \\
	Mehl          & 220 &     g \\
	Butter        & 110 &     g \\
	Sahne         & 125 &     g \\
	Eier          &   5 &
\end{tabular} 

\subsection*{Zubereitung}
\begin{enumerate}
	\item Himbeeren auftauen lassen. Zitronen heiß waschen und trocken reiben. Von 3 Zitronen die Schale fein abreiben. Mehl, 50 g Zucker und 1/4 der Zitronenschale mischen. Butter in Flöckchen, Eigelb und 2–3 EL kaltes Wasser erst mit den Knethaken des Handrührgerätes, dann mit den Händen zu einem glatten Teig verarbeiten. Zugedeckt 30 Minuten stehen lassen. Tarteform in den Kühlschrank stellen.
	\item Teig auf einer bemehlten Arbeitsfläche rund ausrollen. Eine gefettete, mit Mehl ausgestäubte Tarte-Form damit auslegen. Dabei den Rand hochziehen und andrücken. Tarteboden mit einer Gabel mehrmals einstechen. Bei Umluft 200 Grad 12 -- 15 backen. 
	\item Abgeriebene Zitronen halbieren, Saft auspressen. Vanilleschote längs aufschneiden, Mark herausschaben. Vanillemark, 200 g Puderzucker und Eier mit den Schneebesen des Handrührgerätes dick-cremig aufschlagen. Sahne steif schlagen, mit Zitronensaft und übriger Zitronenschale unter die Eier heben.
	\item Backofentemperatur auf 150 Grad herunter schalten. Vorgebackenen Teig mit Paniermehl bestreuen. Eier-Creme in die Form geben. Im  Backofen 40–50 Minuten backen, bis die Creme gestockt ist. 
	\item 3 Zitronen in sehr dünne Scheiben schneiden. 100 ml Wasser und 100 g Zucker in einem Topf aufkochen. Zitronenscheiben darin bei schwacher Hitze 2–3 Minuten glasig dünsten. Zitronenscheiben im Sud erkalten lassen.
	\item Himbeeren und 4 EL Zucker pürieren, durch ein Sieb streichen.
	\item  Tarte herausnehmen und auf einem Kuchengitter auskühlen lassen. Vor dem Servieren die Tarte mit den Zitronenscheiben belegen, mit 50 g Puderzucker bestäuben und unter dem heißen Grill des Backofens goldbraun karamellisieren. Tarte mit Himbeersoße servieren.
\end{enumerate}


\section{Flockensahne} \index{Brandteig}\index{Windbeutel!Torte}

\begin{tabular}{lrr}
	Personen         &                12 &  \\
	Zubereitungszeit &               60 & Minuten \\
	Gesamtzeit       &                3   & Stunden \\
	Entnommen aus    & Chefkoch.de &
\end{tabular} 

\subsection*{Zutaten}
\textbf{Für den Mürbeteig:}\\
\begin{tabular}{lrl}
	Mehl          & 200 &     g \\
	Zucker        &  60 &  \\
	Vanillezucker &   1 &  Pck. \\
	Butter        & 120 &     g \\
	Eier          &   1 &  \\
	Salz          &   1 & Prise \\
	Speisestärke  &  15 &     g
\end{tabular} 

\textbf{Für den Brandteig:}\\
\begin{tabular}{lrl}
	Mehl          & 250 &         g   \\
	Backpulver    &   2 &      Msp.   \\
	Vanillezucker &   2 &      Pck.   \\
	Butter        & 140 &         g   \\
	Wasser        & 500 & ml\\
	Eier &  & 6
\end{tabular} 

\textbf{Für die Füllung:}\\
\begin{tabular}{lrl}
	Sahnesteif    &    5 & Pck.Msp. \\
	Vanillezucker &    5 &     Pck. \\
	Sahne         & 1000 &       ml \\
	Preiselbeeren &    1 &     Glas \\
	Himbeeren     &      &
\end{tabular} 


\subsection*{Zubereitung}
\begin{enumerate}
	\item Aus den Zutaten für den Mürbeteig einen Teig zubereiten und ca. 1 Std. kalt stellen.
	Einen Backrahmen zusammenstecken ( ca.35x25cm ). Darin den Mürbeteig ausrollen und mit einer Gabel mehrmals einstechen. Bei 180\degree\  ca. 10-15 Min. backen.
	
	\item Für den Brandteig das Wasser mit der Butter und dem Vanillezucker aufkochen, das Mehl mit dem Backpulver dazugeben und so lange rühren, bis sich ein Teigklumpen gebildet hat. Etwas abkühlen lassen.\\
	Danach jedes Ei einzeln unterrühren, bis sich eine zähe "Pampe" gebildet hat.
	
	\item Zwei Backbleche mit Backpapier auslegen und den Teig auf die beiden Bleche verteilen und dünn verstreichen. Die Brandteigböden sollen größer werden als der Mürbeteigboden, da man später noch etwas zum obenauf streuen benötigt. Die Böden nacheinander bei 200\degree\  ca. 35 min backen. Den Backofen zwischendurch nicht öffnen, sonst fällt der Teig zusammen.\\
	Mit Hilfe des Backrahmens jeden Brandteigboden in der Größe des Mürbeteigbodens ausstechen und die Reste des Brandteiges aufheben.
	
	\item Den Mürbeteigboden in den Backrahmen legen und mit Konfitüre bestreichen. Ich nehme immer pürierte Preiselbeeren aus dem Glas. Dann eine Brandteigplatte auflegen. Mit einem Teelöffel kleine Häufchen Preiselbeeren auf dem Boden verteilen.
	
	\item Die Sahne mit dem Sahnesteif und dem Vanillezucker steif schlagen. Die Hälfte davon auf den Boden geben und verteilen. Dann den 2. Brandteigboden auflegen. Wieder Preiselbeerhäufchen verteilen und die restliche Sahne darauf verteilen und glatt streichen. Die Reste der beiden Brandteigböden fein auseinander zupfen und wie Flocken auf dem Kuchen verteilen.
	
	\item Den Kuchen zum Schluss mit Puderzucker bestäuben.
	\item Den Kuchen mindestens 2 Std. kalt stellen. Ich habe diesen Kuchen auch schon einen Tag vorher zubereitet. Das tut dem Geschmack keinen Abbruch. 
\end{enumerate}

\section{Windbeutel} \index{Brandteig}\index{Windbeuteltorte}

\subsection*{Zutaten}

\begin{itemize}
	\item 50 g Butter
	\item 1/2 TL Salz
	\item 	150 g Mehl (Type 405)
	\item 	4 M Eier
	\item 	400 g Sahne 
	\item 	1-2 EL Zucker 
	\item 	2 Päckchen Bourbon-Vanillezucker 
	\item 	Backpapier für das Blech 
	\item 	Puderzucker zum Bestreuen
\end{itemize}

\subsection*{Zubereitung}

Für die Brandmasse die Butter in Stückchen mit 250 ml Wasser und Salz in einem Topf aufkochen. Sobald sie geschmolzen ist, das Mehl auf einmal hinein schütten. Mit einem Kochlöffel kräftig rühren, bis sich ein Kloß gebildet hat und am Topfboden ein weißer Belag zu sehen ist. Den Teig in eine Rührschüssel geben. Ein Ei nach dem anderen gründlich mit den Knethaken des Handrührgeräts unter den warmen Teig rühren. 

Den Backofen auf 200° (Umluft 180°) vorheizen. Mit großem Abstand zwölf Rosetten auf ein Blech mit Backpapier mittels zweier Löffel setzen. Das Blech in den Ofen (Mitte) schieben, ein mit Wasser gefülltes zweites Backblech auf der untersten Schiene und rasch die Backofentür schließen. Die Windbeutel in 20-25 Min. goldbraun backen (dabei die Ofentür nicht öffnen!). 

Herausnehmen, von jedem einen Deckel abschneiden und auf einem Kuchengitter auskühlen lassen.

Für die Füllung die Sahne mit Zucker und Vanillezucker steif schlagen und in einen Spritzbeutel mit großer Sterntülle füllen. Die unteren Hälften der Windbeutel mit der Vanillesahne füllen, die Deckel aufsetzen und das Gebäck mit Puderzucker bestreuen.



\section{Käsekuchen mit Mohn} \index{Mürbeteig}\index{Kuchen!Käse}\index{Mohn}
\subsection*{Zutaten}
Für 26\,cm / 28\,cm Springform 

\subsubsection*{Für die Mohnmasse}
\begin{tabular}{lrl}
    Milch          & 250 & ml \\
    Butter         &  60 &  g \\
    Zucker         &  60 &  g \\
    Mohn           & 250 &  g \\
    Zitronenabrieb &   1 & TL
\end{tabular} 
\subsubsection*{Für den Mürbeteig}
\begin{tabular}{lrl}
    Mehl 550     &  220 /260 &             g \\
    Ei           &         1 & mittel / groß \\
    Butter, kalt & 110 / 130 &             g \\
    Zucker       &   70 / 80 &            EL \\
    Backpulver   &       0,5 &            TL
\end{tabular} 

\subsubsection*{Für die Quarkcreme}
\begin{tabular}{lrl}
    weiche Butter  &   80 / 95 &             g \\
    Zucker         & 100 / 115 &             g \\
    Eier           &         2 & mittel / groß \\
    Quark          &  500 /580 &             g \\
    Schmand        & 200  /230 &             g \\
    Zitronenabrieb &         2 &            TL \\
    Zitronensaft   &         2 &            EL \\
    Speisestärke   &    25 /30 &             g
\end{tabular} 

\begin{enumerate}
    \item Milch, Butter, Zucker und Zitronenabrieb in einem kleinen Topf aufkochen. Mohn einrühren und 1-2 Minuten auf niedriger
    Stufe einköcheln lassen. Vom Herd nehmen. Deckel aufsetzen und quellen lassen, bis die Mohnmasse gebraucht wird.
    \item Für den Teig alle Zutaten miteinander verkneten. Die Springform leicht fetten. Teig ausrollen, sodass man Boden und ca. 3
    cm hoher Rand formen kann. Teig in die Form legen und kühl stellen.
    \item Backofen auf 180 Grad Ober- und Unterhitze vorheizen. Weiche Butter mit Zucker schaumig schlagen, Eier einzeln
    unterrühren. Quark, Schmand, Speisestärke, Zitronensaft und -abrieb ebenfalls unterrühren.
    \item Die Mohnmasse auf den Teig geben und glatt streichen. Quarkcreme darüber verteilen. Kuchen ca. 45 Minuten backen. Am
    besten im ausgeschalteten Ofen abkühlen lassen.
\end{enumerate}


\section{Pflaumenkuchen mit Mandelstreusel} \index{Hefeteig}\index{Pflaumen}\index{Zwetchgen}
\subsection*{Zutaten}

\begin{tabular}{lrr}
    Mehl           & 175 &  g \\
    frische Hefe   &  10 &  g \\
    lauwarme Milch &  80 & ml \\
    Butter         &  25 &  g \\
    Ei             &   1 &    \\
    Zucker         &   2 & EL \\
    Zwetschgen     & 500 &  g
\end{tabular} 

Für die Streusel:\\
\begin{tabular}{lrr}
    flüssige Butter &  75 &     g \\
    Mehl            & 120 &     g \\
    Zucker          &  60 &     g \\
    Salz            &   1 & Prise \\
    Zimt            &   1 & Prise
\end{tabular} 

\subsection*{Zubereitung}
\begin{enumerate}
	\item Das Mehl in eine Schüssel geben und eine Mulde in die Mitte drücken, die Hefe dort hinein bröseln und mit der lauwarmen Milch und ein wenig Mehl vom Rand der Mulde zu einem Vorteig verrühren. Das Ganze dann zugedeckt an einem schönen warmen Ort 15 Minuten gehen lassen.
	\item Die Butter schmelzen und zusammen mit dem Ei und dem Zucker zu dem Vorteig geben und alles mit einander verkneten. Den Teig nun weitere 15 Minuten zugedeckt gehen lassen.
	\item Derweil die Zwetschgen waschen, entsteinen und in Viertel schneiden.
	\item Sobald der Teig fertig aufgegangen ist, ihn auf einer bemehlten Oberfläche ausrollen und in eine vorbereitete 26 cm Springform geben. Mit etwas Teig einen Rand ausformen und mit einer Gabel mehrfach in den Boden pieksen. Anschließend die Zwetschgenviertel dicht aneinander gepackt auf dem Hefeteig verteilen.
	\item Alle trockenen Zutaten (Mehl, Zucker, Mandeln, Salz und Zimt) in eine Schüssel geben und die flüssige Butter darüber gießen. Nun alles kräftig mit einander verkneten, bis ein krümeliger Teig entsteht.
	\item Die Streusel auf die Zwetschgen geben, so viele wie gewollt und den Kuchen bei 180°C für etwa 40 Minuten backen, bis die Streusel goldbraun sind.
\end{enumerate}

Für ein Blech zwei große Eier und ansonsten die dreifache Menge von allem.

\section{Johannisbeer Quark \textmd{(siehe \cite[172]{OetkerBackenMachtFreude1992})}}\index{Johannisbeeren}

\section{Aprikosen Tarte} \index{Aprikosen}\index{Tarte!Aprikose}

\chapter{Plätzchen}

\section[Cantuccini]{Cantuccini \textmd{(siehe \cite{ChefkochCantuccini})}}
Mandelplätzchen aus der Toscana/Umbrien
\subsection*{Zutaten}

\begin {tabular}{r l}
    350 g & Mandeln \\
    500 g & Mehl \\
    360 g & Zucker \\
    2 TL & Backpulver \\
    4 Pkt. & Vanillezucker \\
    1 Flasche & Bittermandelaroma \\
    2 Priesen & Salz \\
    50 g & Butter, zimmerwarme \\
    4 & Eier
\end{tabular}
\subsection*{Zubereitung}
\begin{enumerate}
    \item  Alle Zutaten (bis auf die Mandeln) zu einem Knetteig verarbeiten: ein klebriger Knetteig!
    \item  Die Mandeln unterkneten. Den Teig mit etwas Mehl zu einer Kugel formen und 30 Minuten kalt stellen.
    \item  Den Teig in 12 Teile schneiden. Aus jedem Teil eine 25 cm lange Rolle formen. Das Backblech mit Backtrennpapier auslegen. Die Rollen im
    Abstand von 8 cm voneinander auflegen.
    \item  Im vorgeheizten Backofen bei 200 °C Ober-/Unterhitze 15 Minuten vorbacken, kalt werden lassen und dann schräg in 1 cm dicke
    Scheiben schneiden.
    \item  Plätzchen mit einer Schnittfläche auf das Backblech legen und noch einmal im Backofen bei gleicher Temperatur 10 Minuten rösten. Die Cantuccini müssen zum Schluss
    goldbraun sein.
    \item  Die Plätzchen auskühlen lassen und in einer geschlossenen Blechdose aufbewahren, sonst werden sie weich. 
\end{enumerate}
