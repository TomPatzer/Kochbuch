\usepackage{scrhack}
\usepackage{amsmath}
\usepackage{iftex}
\ifPDFTeX
	\usepackage[utf8]{inputenc}  % Deutsche Umlaute direkt eingeben
	\usepackage[T1]{fontenc}     % "Saubere" Schriften in der PDF
\else
	\ifXeTeX
		\usepackage{fontspec}
		\usepackage{unicode-math}
		\fontspec{DejaVu Serif}
		\setmainfont{DejaVu Serif} 
		\setsansfont{DejaVu Sans} 
		\setmonofont{DejaVu Sans Mono} 
		\setmathfont[math-style=TeX]{TeX Gyre DejaVu Math}
	\else
		\ifLuaTeX
			\usepackage{fontspec}
			\usepackage{luatextra}
			\usepackage{unicode-math}
			\fontspec{DejaVu Serif}
			\setmainfont{DejaVu Serif} 
			\setsansfont{DejaVu Sans} 
			\setmonofont{DejaVu Sans Mono} 
			\setmathfont[math-style=TeX]{TeX Gyre DejaVu Math}
		\fi
	\fi 				
\fi
\usepackage[ngerman]{babel}
\usepackage[ngerman]{translator}

\usepackage[babel, german=quotes]{csquotes}
\usepackage{gensymb}
\usepackage{textcomp}
\usepackage{hyperref}
\usepackage[]{varioref}
\usepackage[]{cleveref}
\usepackage{graphicx}
\usepackage{float}
\RedeclareSectionCommand[tocnumwidth=2em]{chapter}
\RedeclareSectionCommand[tocindent=3em,tocnumwidth=3.5em]{section}
\usepackage[]{todonotes}
\usepackage[toc, nopostdot, nonumberlist]{glossaries}
\makeglossaries

\usepackage{makeidx}
\makeindex

\setlength{\parindent} {0.0em}
\setlength{\parskip} {1.5ex plus0.5ex minus0.5ex}

\usepackage[backend=biber, %% Hilfsprogramm "biber" (statt "biblatex" oder "bibtex")
style=authoryear, %% Zitierstil (siehe Dokumentation)
natbib=true, %% Bereitstellen von natbib-kompatiblen Zitierkommandos
hyperref=true, %% hyperref-Paket verwenden, um Links zu erstellen
]{biblatex}
%\usepackage[backend=biber,style=alphabetic]{biblatex}
% die bib-Datei laden:
\addbibresource{Literatur.bib} %



