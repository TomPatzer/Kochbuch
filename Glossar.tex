\glsaddall

\newglossaryentry{Auffrischen}{
    name={Auffrischen},
    description={Das Anstellgut wird durch eine regelmäßige „Auffrischung“ lebendig und fit gehalten. Dieser Prozess wird auch häufig als Füttern bezeichnet. Bei der Auffrischung wird das Anstellgut mit frischem Mehl und Wasser gemischt, um den Mikroorganismen neue Nahrung zu liefern und ihr Wachstum zu fördern. Dieser Vorgang sollte regelmäßig und unter möglichst konstanten Bedingungen durchgeführt werden, um eine gute Qualität und Aktivität des Anstellguts zu gewährleisten.
}}

\newglossaryentry{Autolyse}{
    name={Autolyse},
    description={Siehe auch Nullteig. Die Autolyse bzw. der Autolyseteig wird vor der eigentlichen Teigzubereitung aus Mehl und Wasser oder an-
        deren Schüttflüssigkeiten gemischt und ruht meist abgedeckt für 20 bis 60 Minuten – manchmal auch bis zu 2 Stunden oder sogar über Nacht im Kühlschrank. Dabei verquellen Wasser, Stärke und Eiweißbestandteile aus dem Mehl. Während der Ruhezeit verkettet sich das Klebereiweiß (Gluten), und es bilden sich Glutenstränge (Teigstruktur). Dadurch kann die Knetzeit deutlich reduziert und einer zu starken Erwärmung beim Knetprozess vorgebeugt werden.
}}

\newglossaryentry{Backen}{
        name={Backen},
        description={Das Backen löst im Teig komplexe physikalische und chemische Prozesse aus. Je nach der Temperatur an der Teiglingsoberfläche und im Inneren des Teiglings beginnen unterschiedliche Vorgänge, bei denen sich Krume und Kruste bilden. Die jeweiligen Prozesse laufen nicht gleichmäßig im gesamten Teigling ab, sondern wandern von außen nach innen. Während zum Beispiel in den Randbereichen des Brotes schon alle Mikroorganismen durch die Hitze abgetötet sind, können sie im Inneren weiter für Ofentrieb sorgen.\\
        Quelle: \cite{PloetzblogLexikon2023}
        }}

\newglossaryentry{Ballengare}{
    name={Ballengare},
    description={Die sogenannte Ballengare oder Zwischengare dient als kurze Entspannungsphase für die Kleberstränge. Der Teig wird dabei zunächst locker vorgeformt (meinst rund) und ruht dann abgedeckt (z. B. mit einem Geschirrtuch) zwischen 5–20 Minuten. Dadurch lassen sich manche Teige besser formen oder reißen beim späteren Formen nicht. Anschließend folgt die endgültige Formgebung des Teiges.
}}

\newglossaryentry{Bassinage}{
    name={Bassinage},
    description={Verfahren, ein Teil des Schüttwassers im Knetprozess von Weizenteigen erst am Ende der Knetzeit schluckweise zuzugeben, wenn man bereits eine gute Kleberentwicklung erreicht hat. Dieses Verfahren wird insbesondere bei Teigen mit einem hohen Wassergehalt = hohe Teigausbeute genutzt.
}}

\newglossaryentry{Bruehstueck}{
    name={Brühstück},
    description={Das Brühstück gehört zur Gruppe der Nullteige innerhalb der Vorstufen. Es dient der Verquellung gröberer Brotbestandteile (z.B. Körner, Saaten, Schrote), um den Kaueindruck und die Frischhaltung zu verbessern (siehe auch Quellstück und Kochstück).\newline       
    Für ein Brühstück werden die festen Bestandteile im Verhältnis von ca. 1 : 1 bis 1 : 3 mit kochendem Wasser vermischt und mindestens 2-6 Stunden quellen gelassen. Eine noch optimalere und im Hobbybäckerbetrieb zeitlich passendere Variante ist das Verquellen über 8-12 Stunden bei 6-8°C im Kühlschrank (nachdem das Brühstück ausgekühlt ist). Um enzymatischen Abbau und Fremdgärung zu verhindern, kann die Salzmenge des Hauptteiges mit in das Brühstück eingerührt werden.\newline
    Würden die groben Bestandteile nicht verquollen, würde der Wassergehalt im Teig sinken und der Teig durch Nachquellung zunehmend fester und trockener werden. Üblicherweise sollte die im Brühstück zu verquellende Schrotmenge nicht mehr als 30-50\% der Gesamtmenge der Getreideerzeugnisse ausmachen, da durch das heiße Wasser bereits Stärke verkleistert und dem enzymatischen Abbau stärker ausgesetzt ist.\newline 
    Neben Schrot kann z.B. auch getrocknetes und gemahlenes Brot überbrüht werden. Dieses Altbrot bindet etwa die dreifache Menge seines Eigengewichtes an Wasser.\newline
    Quelle: \cite{PloetzblogLexikon2023} 
}}

\newglossaryentry{DehnenUndFalten}{
        name={Dehnen und Falten},
        description={Das Dehnen und Falten von Teig ist ein Vorgang, bei dem weizendominierten Teigen durch mehrfache Dehnung und Faltung mehr Struktur verliehen wird. Das Klebergerüst wird damit schonend entwickelt. Das Gashaltevermögen steigt. Außerdem dient es der Entgasung und Sauerstoffzufuhr, der Homogenisierung der Teigtemperatur und damit der Unterstützung der mikrobiellen Aktivität.\newline
        Quelle: \cite{PloetzblogLexikon2023} 
        }}

\newglossaryentry{Fermentolyse}{
    name={Fermentolyse},
    description={Das Pendant zur \Gls{Autolyse}. Im Gegensatz dazu ist aber bereits ein Triebmittel (Sauer teig/Vor teig) enthalten. Dadurch verkürzt sich das Vorquellen deutlich auf maximal 20 bis 40 Minuten. Dabei verquellen Wasser, Stärke und Eiweißbestandteile aus dem Mehl. Während der Ruhezeit verkettet sich das Klebereiweiß (Gluten), und es bilden sich Glutenstränge (Teigstruktur). Dadurch kann die Knetzeit deutlich reduziert und einer zu starken Erwärmung beim Knetprozess vorgebeugt werden.
}}


\newglossaryentry{Formen}{
    name={Formen},
    description={Nach der Stockgare und vor der nachfolgenden Stückgare wird der Teig geformt. Der Teig kann rund geformt werden (rundwirken), länglich (langwirken) oder auch in allerlei andere Formen gebracht werden. Dies erfolgt überwiegend auf der bemehlten und selten auf der mit Wasser benetzten Arbeitsfläche.\\
    Ziel ist in aller Regel dabei, dass der Teig nicht nur in Form gebracht wird, sondern an der Teigoberfläche auch eine gewisse Spannung aufbaut. Der Teig sollte dabei aber nur so weit gestrafft werden, dass er nicht reißt.       
}}


\newglossaryentry{Hauptteig}{
        name={Hauptteig},
        description={Der Teig, in dem alle Bestandteile des Brotes enthalten sind.
        }}

\newglossaryentry{LievitoMadre}{
    name={Lievito Madre},
    description={Lievito Madre ist Italiens Antwort auf den traditionellen Sauerteig. Mit seiner festen Konsistenz und dem milden Geschmack ist er ein unverzichtbares Element in der italienischen Brotkultur. Ein Lievito Madre ist ein spezieller italienischer Weizensauerteig, der fest geführt wird. Er ist mild und triebstark und auch für süße Backwaren geeignet.         
}}

\newglossaryentry{Quellstueck}{
    name={Quellstück},
    description={Ein Quellstück ist eine wichtige Komponente beim Brot backen, insbesondere wenn das Rezept gröbere oder trockenere Zutaten wie Altbrot, Samen, Schrot oder Trockenfrüchte enthält. Dieser sogenannte Nullteig oder Vorstufe hilft dabei, diese trockenen Zutaten vorzuquellen, indem sie in einem Verhältnis von etwa 1:1 bis 1:2 mit kühlem oder lauwarmem Wasser übergossen werden.
}}

\newglossaryentry{Sauerteig}{
        name={Sauerteig},
        description={In einem Sauerteig entwickeln sich homofermentative (milchsäurebildende) und
        heterofermentative (essigsäurebildende) Milchsäurebakterien (letztere werden oftmals fälschlicherweise als Essigsäurebakterien bezeichnet), außerdem Hefen. 
        Die heterofermentativen Bakterien erzeugen außerdem Kohlenstoffdioxid, das gemeinsam mit dem Kohlenstoffdioxid der alkoholischen Hefegärung das Gärgas bildet und für den Trieb sorgt.
        }}

\newglossaryentry{Stueckgare}{
    name={Stückgare},
    description={
        Die letzte Reifezeit des geformten Teiges vor dem Backen. Die Stückgare er folgt entweder mit Schluss nach oben oder nach unten in einem Gärkörbchen, Bäckerleinen oder in einer Backform.
        Die Stückgare kann je nach Rezept bei Raumtemperatur, an einem
        warmen Ort oder auch kalt erfolgen.
}}
\newglossaryentry{Stockgare}{
    name={Stockgare},
    description={Zeit die man dem Teig nach dem Kneten lässt um sich zu entwickeln. Weiche Weizenteige werden oft während der Stockgare gedehnt und gefaltet um die Teigstruktur zu verbessern. Zur Stockgare wird der Teig in einer Schüssel oder Wanne luftdicht abgedeckt damit er nicht austrocknet
}}

\newglossaryentry{Vorheizen}{
        name={Vorheizen},
        description={Der Ofen sollte bei Ober/Unterhitze ausreichend lange vorgeheizt werden, um das perfekte Brot zu backen. Umluft eignet sich weniger gut, da diese Einstellung die Oberfläche des Brotes zu schnell austrocknen würde.
        }}

\newglossaryentry{Zwischengare}{
    name={Zwischengare},
    description={Die Zwischengare ist eine sehr kurze (ca. 5 – 30 Minuten) Ruhephase zwischen Arbeitsvorgängen. Häufig wird die Zwischengare genutzt, um die Kleberstränge rundgewirkter Teiglinge kurz entspannen zu lassen. Andernfalls würde die Teigoberfläche beim weiteren Wirken reißen. \cite{PloetzblogLexikon2023}
}}




%\newglossaryentry{ }{
    %    name={},
    %    description={
        %}}


%
%
%
%Beschreibung:
%Das Brühstück gehört zur Gruppe der Nullteige innerhalb der Vorstufen. Es dient der Verquellung gröberer Brotbestandteile (z.B. Körner, Saaten, Schrote), um den Kaueindruck und die Frischhaltung zu verbessern (siehe auch Quellstück und Kochstück).
%
%Für ein Brühstück werden die festen Bestandteile im Verhältnis von ca. 1 : 1 bis 1 : 3 mit kochendem Wasser vermischt und mindestens 2-6 Stunden quellen gelassen. Eine noch optimalere und im Hobbybäckerbetrieb zeitlich passendere Variante ist das Verquellen über 8-12 Stunden bei 6-8°C im Kühlschrank (nachdem das Brühstück ausgekühlt ist). Um enzymatischen Abbau und Fremdgärung zu verhindern, kann die Salzmenge des Hauptteiges mit in das Brühstück eingerührt werden.
%
%Würden die groben Bestandteile nicht verquollen, würde der Wassergehalt im Teig sinken und der Teig durch Nachquellung zunehmend fester und trockener werden. Üblicherweise sollte die im Brühstück zu verquellende Schrotmenge nicht mehr als 30-50% der Gesamtmenge der Getreideerzeugnisse ausmachen, da durch das heiße Wasser bereits Stärke verkleistert und dem enzymatischen Abbau stärker ausgesetzt ist.
%
%Neben Schrot kann z.B. auch getrocknetes und gemahlenes Brot überbrüht werden. Dieses Altbrot bindet etwa die dreifache Menge seines Eigengewichtes an Wasser.
%
%Quellen:
%Steffen, Lutz Geißler
%/home/thomas/Sandbox/Privat/Privat/Text/Kochbuch/Glossar.tex
%A
%
%Anstellgut (ASG) : Als Anstellgut bezeichnet man die Sauerteigkultur, die man zum Ansetzen eines neuen Sauerteigs benutzt. Ich lagere meine Sauerteigkultur = Anstellgut im Marmeladenglas im Kühlschrank für bis zu 3 Wochen. Dann muss die Kultur aufgefrischt werden, d.h. 50 g Mehl/50 g Wasser/10 g alte Kultur = Anstellgut für ca. 16-20 h von 30° C auf Raumtemperatur fallend stehen lassen, danach im Kühlschrank lagern. Grundsätzlich kann man einen Sauerteig durch Auffrischen mit einer anderen Mehlsorte umzüchten, also z.B. von Roggen- nach Weizensauerteig. Teilweise wird das Anstellgut auch als Starter bezeichnet.
%
%Auffrischen: Als Auffrischen bezeichnet der Bäcker das Füttern der Sauerteigkultur mit frischem Mehl und Wasser, damit die Mikroorganismen frische Nahrung bekommen und sich wieder vermehren können. Übliche Verhältnisse zum Auffrischen von weichem Sauerteig ist ein Verhältnis 1:5:5 oder 1:10:10 von Anstellgut/Wasser/Mehl.
%
%Autolyse: Als Autolyse bezeichnet man das Quellenlassen von Mehl mit Wasser für 20-60 min ohne die Zugabe von Salz. Dabei quellen die Stärke und die Eiweiße im Mehl auf, was die Backeigenschaften verbessert und das Kneten beschleunigt.
%B
%
%Bassinage: Verfahren, ein Teil des Schüttwassers im Knetprozess von Weizenteigen erst am Ende der Knetzeit schluckweise zuzugeben, wenn man bereits eine gute Kleberentwicklung erreicht hat. Dieses Verfahren wird insbesondere bei Teigen mit einem hohen Wassergehalt = hohe Teigausbeute genutzt.
%D
%
%Dehnen und Falten (oft auch  “stretch and fold” oder s+f genannt)
%Weiche Weizenteige können über das wiederholte dehnen und falten schonend eine gute Struktur bekommen. Dazu wird, vorzugsweise  in einer geölten Wanne oder Schüssel, mit den feuchten Händen eine Seite des Teigs nach oben gezogen und dann über den Teig gefaltet , diesen Vorgang wiederholt man  auf den anderen 3 Seiten. Nach 20 -30 min kann man dann eine weitere Runde starten. Man merkt dabei, dass der Teig wesentlich mehr Struktur bekommt und straffer wird.
%E
%
%Einschießer/Einschießen
%Als Einschießen bezeichnet man den Vorgang den Teigling  mit einem Schieber oder Einschießer in den  Ofen zu bringen. Als Hobbybäcker benutzt man dafür meist ein passendes Brett und verwendet als Trennmittel Gries, feinen Schrot oder einfach Backpapier.
%
%Entgasen/Ausstoßen
%Wenn ein Teig nach der Stockgare gut aufgegangen ist, drückt man ihn mit der flachen Hand oder der Faust auf der Arbeitsfläche platt und sorgt damit dafür, dass sich große Gärblasen verteilen und auch ein Austausch von Kohlendioxid gegen Luftsauerstoff stattfindet was die weitere Entwicklung der Hefen fördert.
%F
%
%Fenstertest: Der Fenstertest dient zur Beurteilung der Kleberentwicklung eines weizenlastigen Teiges. Dazu zieht man den gekneteten Teig zwischen den feuchten Fingern so dünn, dass Licht durchscheint. Reißt der Teig davor ist er noch nicht ausreichend geknetet.
%
%Fingertest: Mit dem Fingertest prüft man die Gare eines Teiglings während der Stückgare: Dazu drückt man leicht mit dem Finger auf den Teigling und kann mit etwas Erfahrung die folgenden Garezustände unterscheiden:
%
%Untergare: Abdruck springt schnell und komplett wieder zurück
%knappe Gare: Abdruck geht langsam und fast  komplett zurück
%Volle Gare: Abdruck geht nur noch wenig zurück es bleibt eine leichte Delle
%Übergare: Abdruck bleibt unverändert oder Teigling fällt sogar zusammen
%
%G
%
%Glutenentwicklung: (Kleberentwicklung) Beim Kneten des Teigs vernetzt sich das Gluten (Kleber) immer stärker, um ein großvolumiges Brot zu erhalten ist eine gute Glutenentwicklung notwendig, diese kann durch den sog. Fenstertest überprüft werden.
%H
%
%Hefe: Hefe ist neben Sauerteig das wichtigste Triebmittel in allen Brotteigen. Man kann Trocken- oder Frischhefe einsetzen. Alle Angaben in meinen Rezepten beziehen sich auf Frischhefe.
%Umrechnungsregel: Trockenhefe in Frischhefe: 1 g Trockenhefe entspricht 3 g Frischhefe.
%Als Hilfe für das Portionieren kleiner Hefemengen gibt es eine sog. Hefeschablone, auch käuflich bei der Draxmühle zu erwerben.
%L
%
%Lievito Madre (LM): LM ist ein sehr milder und triebstarker Weizensauerteig italienischem Ursprungs. Er wird oft für helle Brote, Brötchen und auch für Pannetone verwendet. LM wird recht fest, mit einer Teigausbeute von 150 geführt, d.h. auf 100 g Mehl 50 g Wasser und führt zu einem starken Ofentrieb.
%
%Aufgefrischt wird LM normalerweise so: 1 Teil alter LM, 1 Teil Weizenmehl, 0,5 Teile Wasser (35 -40°C) zu einem festen Teig verkneten, kreuzwesie einschneiden und abgedeckt 4 h bei 30°C gehen lassen.  Dabei soltle er sich mindestens verdoppeln. Dann ist er zum Backen bereit. Nicht vergessen etwas LM abzuzweigen und als “Starterkultur” im Kühlschrank aufbewahren.
%
%Langwirken: Als Langwirken bezeichnet man das Formen eines länglichen Brotlaibs. Man startet immer in dem man den Teig rundwirkt (siehe unten) und ihn. dann nach kurzer Entspannungspause in eine längliche Form bringt. Dafür gibt es diverse Techniken, z.B. hier im Video erklärt.
%P
%
%Poolish: Flüssiger Weizenvortieg mit Hefe: Weizenmehl wird 1:1 mit Wasser gemischt und mit wenig Hefe  (0,2-0,3 %) versetzt und 12-14 h bei Raumtemperatur oder im Kühlschrank stehen lassen.
%R
%
%Rundschleifen: Nennt der Bäcker den Vorgang zum Formen runder Brötchen, es gibt diverse Filmchen dazu bei YouTube
%
%Rundwirken:  So nennet man den Vorgang aus einem Teig einen runden Laib zu formen und dabei im Teig Spannung aufzubauen.  Am besten schaut mich das in einem Video auf YouTube an
%S
%
%Schluß: Die Seite eines Teiglings auf der der Teig eingefaltet wird um Spannung im Teigling zu erzeugen, wird als Schluß bezeichnet. Backt man ein Brot oder Brötchen mit dem Schluß nach oben, reisst die Oberfläche rustikal auf und es muss nicht eingeschnitten werden.
%
%Schwaden: Bäckerdeutsch für  Dampf, der im Ofen erzeugt wird, dies ist insbesondere in der ersten Backphase wichtig damit, ein guter Ofentrieb und eine schöne rösche Kruste erhalten wird. Später wird der Dampf meist abgelassen, damit die Kruste kross wird.
%
%Stippen: So nennt der Bäcker den Vorgang in die Oberfläche eines Teiglinge vor dem Backen kleine Löcher zu machen, dafür gibt es spezielle Stippwalzen/Stipproller/Teigigel oder man behilft sich z.B. mit einem Schaschlikspieß oder ähnlichem. Die dient dazu bei einem fast vollgaren Brot das aufreißen der Kruste zu verhindern.
%
%t.
%
%Stückgare: Zeit in dem sich der Teigling nach dem Formen entwickelt und die richtige Gare vor dem Backen erreicht. Wichtig ist es die Teiglinge mit einem Küchentuch und Folie abzudecken, damit sie nicht austrocknen.
%T
%
%Teigausbeute (TA): Die Teigausbeute gibt das Verhältnis von Mehl zu Flüssigkeit im Teig an, eine hohe Teigausbeute bedeutet, dass viel Flüssigkeit im Teig vorhanden ist. Berechnet wird sie folgendermaßen: Teigausbeute = Flüssigkeitsmenge/Mehlmenge*100+100,
%also z.B. 650g  Wasser /1000g Mehl ergeben eine TA von 165.
%W
%
%Wirken:  Mit dem Begriff “wirken” bezeichnet der Bäcker das in Form bringen von Teiglingen mit speziellen Techniken, welche Spannung im Teigling erzeugen und für die Krumenstruktur und das Volumen des Brots eine sehr wichtige Rolle spielen. Man unterscheidet insbesondere das Rundwirken von runden Laiben und das Langwirken von länglichen Laiben.
