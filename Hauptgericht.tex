\chapter{Hauptgerichte}
\section{Vegetarisch}
\subsection{Grünkernbratlinge Ritterburger} \index{vegetarisch}\index{Grünkernbratlinge}\index{Grünkernbratlinge!Ritterburger}\index{Vollwert!Grünkernbratlinge}\index{Verwertung!Käse}
\subsubsection*{Allgemeines}
\begin{tabular}{lrl}
    Personen         &                        8 &  \\
    Zubereitungszeit &                       45 & Minuten \\
    Gesamtzeit       &                       45 & Minuten \\
    %	Entnommen aus    &  &
\end{tabular} 

\subsubsection*{Zutaten}
\begin{tabular}{rll}
    500 & g  & Grünkern          \\
    0,5 & l  & Wasser            \\
    2 & TL & Gemüsebrühe       \\
    2 &    & Zwiebeln          \\
    2 &    & Eier              \\
    2  & EL & Senf              \\
    200 & g  & Restkäse          \\
    &    & Salz              \\
    &    & schwarzer Pfeffer \\
    &    & Schnittlauch      \\
    &    & Petersilie        \\
    &    & Olivenöl          \\
    &    & Sesam
\end{tabular} 
\subsubsection*{Zubereitung}
\begin{enumerate}
    \item Grünkern mahlen. 400~g Stufe 4 und 100~g Stufe 8.
    \item Grünkern mit Gemüsebrühe 10 Minuten quellen lassen.
    \item Die Zwiebeln schälen und in feine Würfel schneiden. Kräuter kleinschneiden, mit den Eier, dem geriebenen Käse, dem Senf, den Zwiebeln und der Grünkernmasse vermengen.
    \item Bratlinge formen und im Sesam wälzen.
    \item Bei schwacher Hitze (Induktionsherd 5 -6) braten.
\end{enumerate}


\subsection{Pilz-Lasagne}\index{Lasagne}\index{Pilze}\index{vegetarisch}

\subsubsection*{Zutaten} 
Zutaten für eine rechteckige Auflaufform von ca. 25 x 35cm, also so groß, dass vier Lasagne-Nudelplatten zum Auslegen reichen, 4 Personen

\paragraph*{Für die Pilzmischung}

250g frische Shiitake-Pilze \index{Shiitake Pilze} \\
250g Kräuterseitlinge\index{Kräuterseitlinge} \\
1 Pfund braune Champignons \index{Champignons} \\
2 Zwiebeln \\
1 große Knoblauchzehe \\
300ml Sahne \\
1 dickes Bund Schnittlauch \\
Pflanzenöl, Pfeffer, Zucker, Muskat \\
50g geriebener Parmesan

\paragraph*{Für die Bechamelsauce}
50g Butter\\
50g Mehl\\
500ml Milch\\
3 Eier\\
100g geriebener Bergkäse\\
Salz, Pfeffer, Zitronensaft, Muskat

\paragraph*{Außerdem}

1 Packung Lasagne-Nudelblätter (1 Pfund)

\subsubsection*{Zubereitung}

Die Champignons putzen, Stängel unten abschneiden und relativ klein würfeln. Zwiebeln längs halbieren, quer in dünne Streifen und Schnittlauch in kleine Röllchen schneiden. Knoblauch fein hacken.

Von den Shiitake die Stängel abknipsen, die Hüte in schmale Streifen schneiden. Die Kräuterseitlinge etwas größer als die Champignons würfeln.
Nun die Zwiebeln in einem großen Topf in etwas Öl glasig dünsten, Temperatur hoch schalten und die Champignons kräftig anbraten. Salzen, pfeffern und mit dem Knoblauch würzen.

In einer anderen Pfanne zunächst die Kräuterseitlinge kräftig braten bis sie angebräunt sind, etwas salzen und zu den Champignons geben. In derselben Pfanne mit den Shiitake-Pilzen genauso verfahren und sie dann mit der Sahne unter die Champignons rühren. Alles fünf Minuten offen köcheln, zum Schluss die Schnittlauchröllchen unterziehen und herzhaft abschmecken

Für die Bechamelsauce bei niedriger Hitze in einem Topf die Butter zerlaufen lassen, das Mehl einstreuen und rühren, bis das Fett ganz aufgesogen ist. Dann die heiße Milch portionsweise dazu gießen, und immer solange rühren, bis die Milch jeweils aufgesogen ist. So entsteht langsam eine glatte, cremige Sauce. Wenn die Milch aufgebraucht ist, die Sauce einmal kurz aufkochen, mit ein paar Tropfen Zitronensaft, Salz, Pfeffer und Muskat würzen, den Käse hinein geben  und solange rühren bis die Sauce wieder glatt ist. Die Sauce fünf Minuten offen köcheln und dann etwas abkühlen lassen. Jetzt einzeln die 
Eier hinein schlagen und glatt verrühren, nicht mehr kochen lassen.
Die Auflaufform mit einer Schicht Lasagne-Blätter auslegen, die Hälfte der Pilzmischung darauf verteilen und die Hälfte des Parmesans darüber streuen. Dann wieder eine Lage Nudelblätter, Pilzmischung mit Parmesan darauf und das Ganze wieder mit Nudelblättern abdecken. Zum Schluss die Bechamelsauce darüber gießen, diese im vorgeheizten Backofen bei 200 Grad 45 Minuten backen. Wenn die Oberfläche zu schnell bräunt, die Form mit Alufolie abdecken.

Lasagne heraus nehmen, zehn Minuten stehen lassen, dann in Stücke schneiden und servieren.

\subsection{Grünkern-Lasagne}\index{Lasagne}\index{Grünkern!Lasagne}\index{Lasagne!Grünkern}\index{vegetarisch}

\subsubsection*{Zutaten} 
Zutaten für eine rechteckige Auflaufform von ca. 25 x 35cm, also so groß, dass vier Lasagne-Nudelplatten zum Auslegen reichen, 4 Personen

\paragraph*{Für die Bolognese}

80 g 	Grünkern, geschroteter\\
400 ml 	Gemüsebrühe\\
2  	Karotte(n)\\
2 Stangen Staudensellerie\\
1  	Zwiebel(n)\\
800 g 	Tomate(n), gehackte\\
1 Lorbeerblatt\\
Tymian und Rosmarin\\
2 EL 	Tomatenmark\\
etwas 	Rotwein bei Bedarf\\
2 EL, gestr. Basilikum, gehackt\\
Salz\\
Pfeffer\\
etwas 	Olivenöl 

\paragraph*{Für die Bechamelsauce}
50g Butter\\
50g Mehl\\
500ml Milch\\
3 Eier\\
100g geriebener Bergkäse\\
Salz, Pfeffer, Zitronensaft, Muskat

\paragraph*{Außerdem}

1 Packung Lasagne-Nudelblätter (1 Pfund)

\subsubsection*{Zubereitung}

Den Grünkern grob mahlen (6-8). Den Grünkernschrot in einem Topf so lange anrösten, bis er ein wenig dunkler wird. Mit der Gemüsebrühe ablöschen und kurz aufkochen lassen. Den dann Topf von der Platte nehmen und den Grünkernschrot ca. 20 Minuten quellen lassen. 

In der Zwischenzeit das Gemüse waschen, putzen bzw. schälen und in möglichst kleine Würfel schneiden. Die Zwiebel schälen und ebenfalls fein würfeln.

In einer beschichteten Pfanne etwas Olivenöl erhitzen und das Gemüse darin anrösten. Den gequollenen Grünkernschrot zugeben und etwas anbraten. Mit den gehackten Tomaten ablöschen, die Kräuter und evtl. etwas Rotwein hinzugeben. Ca. 30 Minuten auf kleiner Stufe köcheln lassen. Dann das gehackte Basilikum sowie das Tomatenmark unterrühren. Zum Schluss die Soße mit Salz und Pfeffer abschmecken.

Für die Bechamelsauce bei niedriger Hitze in einem Topf die Butter zerlaufen lassen, das Mehl einstreuen und rühren, bis das Fett ganz aufgesogen ist. Dann die heiße Milch portionsweise dazu gießen, und immer solange rühren, bis die Milch jeweils aufgesogen ist. So entsteht langsam eine glatte, cremige Sauce. Wenn die Milch aufgebraucht ist, die Sauce einmal kurz aufkochen, mit ein paar Tropfen Zitronensaft, Salz, Pfeffer und Muskat würzen, den Käse hinein geben  und solange rühren bis die Sauce wieder glatt ist. Die Sauce fünf Minuten offen köcheln und dann etwas abkühlen lassen. Jetzt einzeln die 
Eier hinein schlagen und glatt verrühren, nicht mehr kochen lassen.

Die Auflaufform mit einer Schicht Lasagne-Blätter auslegen, erst Bolognese mit Parmesans darüber streuen. Dann wieder eine Lage Nudelblätter, Béchamesauce das Ganze wieder mit Nudelblättern abdecken, usw. Zum Schluss die Bechamelsauce darüber gießen, diese im vorgeheizten Backofen bei 200 Grad 45 Minuten backen. Wenn die Oberfläche zu schnell bräunt, die Form mit Alufolie abdecken.

Lasagne heraus nehmen, zehn Minuten stehen lassen, dann in Stücke schneiden und servieren.

\subsection{Conchiglie mit Joghurt, Erbsen und Chili} \label{sec:Conchiglie}\index{Conchiglie}\index{Erbsen}

\subsubsection*{Allgemeines}
\begin{tabular}{lrl}
    Personen         &   6 &  \\
    Zubereitungszeit &   ? & Minuten \\
    Woher & &\cite[vgl.][]{OttoLenghiJerusalem}
\end{tabular} 

\subsubsection*{Zutaten}
\begin{tabular}{lrl}
    500 &  g & griechischer Joghurt                                   \\
    150 & ml & Olivenöl                                               \\
    4   &    & Knoblauchzehen (zerdrückt oder ganz klein geschnitten) \\
    500 &  g & Erbsen (TK, aufgetaut und vorbereitet)                 \\
    500 &  g & Conchiglie (große Muschelnudeln)                       \\
    50  &  g & Cashewkerne                                            \\
    1-2 & TL & Chiliflocken (je nach gewünschter Schärfe)             \\
    40  &  g & frisches Basilikum, Blätter zerrissen                  \\
    250 &  g & Feta                                                   \\
    &    & Salz Pfeffer, frisch gemahlen
\end{tabular} 
\subsubsection*{Zubereitung}
\begin{enumerate}
    \item Joghurt mit zwei Dritteln des Olivenöls und 100g der Erbsen und dem Knoblauch pürieren, bzw. zu einer cremigen Masse verrühren). Die Joghurtsauce in einer großen Schale bereitstellen.
    \item Die Nudeln parallel in reichlich Salzwasser kochen.
    \item Nun das restliche Öl mit den Pinienkernen und den Chiliflocken in der Pfanne rösten bis sie goldbraun sind und die Flüssigkeit rötlich gefärbt ist.
    \item Die restlichen Erbsen erhitzen.
    \item Die Nudeln abgießen und Stück für Stück mit der Joghurtsauce in der großen Schale vermischen (nicht zu viel auf einmal, sonst gerinnt der Joghurt).
    \item Das Ganze mit ordentlich Pfeffer \& Salz abschmecken.
    \item Die Mischung auf dem Teller anrichten, Erbsen drübergeben und Feta darauf zerbröseln. Nun Basilikum darauf streuen und die Pinienkerne mit Öl drübergeben.
\end{enumerate}


\subsection{Apfel-Rotkohl geschmort} \index{Rotkohl}

\subsubsection*{Allgemeines}
\begin{tabular}{lrl}
    Personen         &   4 &  \\
    Zubereitungszeit &   50 & Minuten \\
\end{tabular} 

\subsubsection*{Zutaten}
\begin{tabular}{lrl}
    1 &     kg & Rotkohl                       \\
    2 &     ml & Äpfel säuerlich               \\
    1 &  große & Zwiebel                       \\
    1 &     TL & Zucker                        \\
    1 &     EL & Weinessig                     \\
    2 & Tassen & Wasser                        \\
    1 & Tasse1 & Rotwein                       \\
    2 &        & Nelken                        \\
    1 &        & Loorbeerblatt                 \\
    &        & Salz Pfeffer, frisch gemahlen \\
    1 &     EL & Johannisbeermarmelade
\end{tabular} 

\subsubsection*{Zubereitung}
\begin{enumerate}
    \item Die schlechten Blätter vom Kohl entfernen, vierteln und den Strunk herausschneiden.
    \item Die Viertel in dünne streifen hobeln oder klein schneiden.
    \item Die Äpfel schälen entkernen und in dünne Scheibchen schneiden.
    \item Die Zwiebel schälen und in kleine Stücke schneiden.
    \item Das Fett in einem großen Schmortopf schmelzen und die Kohlstreifen sowie die Apfelscheibchen darin andünsten.
    \item Nun den Zucker, das Salz ,den Pfeffer ,den Essig,die Lorbeerblätter und die mit Nelken gespickte Zwiebel dazugeben.
    \item Mit dem Wasser und dem Rotwein auffüllen und den Kohl bei milder Hitze ca 45 Min schmoren lassen
    \item Nach der Garzeit den Kohl mit Johannisbeermarmelade abschmecken
    
\end{enumerate}



\subsection{Gnocchi}\label{sec:Gnocchi}\index{Gnocchi}

\subsubsection*{Allgemeines}
\begin{tabular}{lrl}
    Personen         &  4  &  \\
    Zubereitungszeit &  60 & Minuten \\
\end{tabular} 

\subsubsection*{Zutaten}
\begin{tabular}{lrl}
    1000 &  g &  Kartoffeln \\
    2   &    & Eier               \\
    4   & EL & Parmesan           \\
    4   & EL & Kräuter            \\
    200 &  g & Semola               \\
    &    & Salz, Pfeffer und Muskat
\end{tabular} 


\subsubsection*{Zubereitung}

\begin{enumerate}
    \item Die Kartoffeln kochen etwa 20 Minuten kochen, Backofen auf 100 Grad Umluft vorheizen. Die Kartoffeln abgießen, auf ein Blech legen, mehrfach mit Gabel einstechen und ca 30 Minuten im Ofen trocknen.
    \item Die Kartoffeln, pellen und durch eine Kartoffelpresse in eine Rührschüssel drücken. Eier, Parmesan und die Kräuter unterrühren.  Salzen, pfeffern und Mehl dazugeben. Einige Minuten durchkneten.
    \item  Den Teig in kleine Portionen teilen und daumendicke Rollen formen. 
    \item In siedendem Wasser garen lassen, bis sie an der Oberflächen schwimmen
\end{enumerate}




\subsection{Kartoffelknödel}\label{sec:Kartoffelknoedel}\index{Kartoffelknödel}\index{Knödel}

\subsubsection*{Allgemeines}
\begin{tabular}{lrl}
    Personen         &    &  \\
    Zubereitungszeit &  30 & Minuten \\
\end{tabular} 

\subsubsection*{Zutaten}
\begin{tabular}{lrl}
    1   & kg & Kartoffel(n) \\
    100 &  g & Mehl         \\
    100 &  g & Stärkemehl   \\
    2   &    & Eier         \\
    &    & Salz
\end{tabular} 


\subsubsection*{Zubereitung}

\begin{enumerate}
    \item Die Kartoffeln kochen, abschrecken und pellen. Diese dann noch heiß entweder durch die Kartoffelpresse drücken oder aber einfach mit einem Stampfer zu Brei verarbeiten.
    \item Wenn die Kartoffeln nur noch handwarm sind, Mehl, Stärke und Eier einarbeiten. Den Teig mit Salz abschmecken.
    \item Der Teig sollte nur noch leicht feucht sein. Wer sich wegen der Konsistenz unsicher ist, kann einen Probeknödel kochen und bei Bedarf noch Stärke hinzugeben. Mit nassen Händen kleine Knödel aus dem Teig formen und in reichlich leicht siedendem Salzwasser in ca. 10 Minuten gar ziehen lassen. Steigen die Knödel nach oben, lässt man sie noch ca. 5 Minuten schwach köcheln und kann sie dann herausnehmen und servieren.
\end{enumerate}
Tipp: Den Teig mit Kräutern würzen oder als vegetarische Hauptspeise mit gewürfeltem Käse, angebratenen Pilzen oder Zwiebeln füllen. Dazu eine Sahnesoße reichen. 

\subsection{Gefüllte Paprika im Kartoffelbett} \index{Kartoffelpüree}\index{Paprika}\index{Auflauf}

\subsubsection*{Allgemeines}
\begin{tabular}{lrl}
    Personen         &   4 &  \\
    Zubereitungszeit &   & Minuten \\
\end{tabular} 

\subsubsection*{Zutaten}
\begin{tabular}{lrl}
    3   & große & rote Paprika                        \\
    500 &     g & tiefgefrorener Blattspinat          \\
    1   &       & Zwiebel                             \\
    1   &       & Knoblauchzehe                       \\
    2   &    EL & Sonnenblumenöl                      \\
    100 &    ml & Gemüsebrühe                         \\
    &       & Salz, Pfeffer, geriebene Muskatnuss \\
    250 &     g & Ricotta                             \\
    800 &     g & Kartoffeln                          \\
    &       & Milch                               \\
    50  &     g & Butter                              \\
    80  &     g & Parmesan                            \\
    50  &     g & Cashew Kerne
\end{tabular} 


\begin{figure}
    \centering
    \includegraphics[width=0.7\linewidth]{Bilder/Gefüllte-Paprika-Im-Kartoffelbett}
    \caption{Gefüllte Paprika im Kartoffelbett}
    \label{fig:gefullte-paprika-im-kartoffelbett}
\end{figure}

\subsubsection*{Zubereitung}

\begin{enumerate}
    \item Blattspinat auftauen und gut abtropfen lassen. Paprika halbieren, putzen, waschen und trocken reiben. Zwiebeln schälen, halbieren und fein würfeln. Knoblauch schälen und fein hacken. 
    \item Öl in einem weiten Topf erhitzen. Zwiebeln und Knoblauch darin ca. 2 Minuten andünsten. Blattspinat zufügen mit Brühe ablöschen und weitere ca. 3 Minuten dünsten. Mit Salz, Pfeffer und Muskat würzen. Kurz abkühlen lassen. Ricotta hineingeben und vermengen. Paprikahälften gleichmäßig mit Spinat befüllen, auf ein Backblech geben und im vorgeheizten Backofen (Umluft: 175 °C) ca. 20 Minuten backen. 
    \item Inzwischen Kartoffeln schälen und garen. In einem Topf Milch und Butter erhitzen und die Kartoffeln mit der Milchmischung zu einem Püree stampfen. Mit Salz, Pfeffer und Muskat abschmecken. 
    \item Käse fein reiben. Heißes Kartoffelpüree in eine Feuerfeste Form geben. Paprika oben drauf setzen, leicht andrücken. Mit Käse bestreuen und weitere 8–10 Minuten im Backofen backen. 
    \item Cashew Kerne in einer Pfanne ohne Fett goldbraun rösten. Paprika aus dem Ofen nehmen und mit Pinienkernen garnieren. 
    
    
\end{enumerate}

\subsection{Kartoffel - Wirsing - Auflauf mit Feta}\label{sec:WirstingAuflauf}\index{Kartoffelauflauf}\index{Wirsing}\index{Feta}

\subsubsection*{Zutaten für 4 Portionen}
\begin{tabular}{lrl}
    1                &    & Wirsing                       \\
    800              &  g & Kartoffel(n), festkochende    \\
    Salz und Pfeffer &  &\\
    1                &    & Zwiebel                       \\
    1                &    & Knoblauchzehe                 \\
    2                & EL & Öl                            \\
    150              &  g & Käse, geriebener (Emmentaler) \\
    250              &  g & Feta-Käse                     \\
    250              & ml & Schmand/Sahne zum Kochen
\end{tabular} 

Den Wirsing in Streifen schneiden und in kochendem Salzwasser etwa 3 Minuten blanchieren. Die Kartoffeln schälen, in Scheiben schneiden und ebenfalls etwa 3 Minuten in kochendem Salzwasser blanchieren.

Zwiebel, Knoblauch und Speck fein würfeln und in 1 EL Öl anschwitzen. Den Wirsing hinzugeben und kurz mitbraten.

Eine Auflaufform mit 1 EL Öl einfetten und die Kartoffelscheiben abwechselnd mit Wirsing, zerbröckeltem Feta und Emmentaler schichten. Jede Schicht mit Salz (wenig!) und Pfeffer würzen. Mit einer Schicht Emmentaler abschließen und alles mit der Cremefine oder Sahne übergießen.

Im Backofen bei 180 Grad etwa 35 Minuten backen.



\subsection{Rote Bete mit Zucchini (oder Lauch) und Birnen -pfannengerührt} \label{sec:RoteBete:pfannengerührt}\index{Rote Bete}\index{Rote Bete!pfannengerührt}
\subsubsection*{Allgemeines}
\begin{tabular}{ll}
    Personen         &  4   \\
    Zubereitungszeit &  40 Minuten \\
\end{tabular} 
\subsubsection*{Zutaten}
\begin{tabular}{r l}
    400 g & rote Beete  \\
    400 g & Zucchini    \\
    400 g & feste Birne \\
    400 g & Lauch       \\
    & Olivenöl
\end{tabular}


Rote Bete gut bürsten, ungeschält in kleine, 1/2cm dicke Würfel schneiden. Öl oder Fett in die Pfanne geben, dieses salzen und die Rote Bete-Würfel darin 15 Minuten schmoren (Deckel). Gelegentlich wenden oder schütteln.

Währenddessen die Zucchini 1cm dick Würfeln und nach 15 Minuten zu den Rote Bete in die Pfanne geben.


Nun darf etwas gesalzen werden. Etwas Öl hinzu, wenn es in der Pfanne zu trocken erscheint. Weitere 10 Minuten dünsten lassen. Sollten die Rote Bete noch frische Blätter haben, diese in feine Streifen schneiden und 5 Minuten nach den Zucchini zugeben. Die Birnen ungeschält erst in Achtel, dann in Scheiben schneiden und zum Schluss unter das Gemüse mischen.

Alles zusammen nochmal 5 Minuten dünsten lassen. Nur mit Salz abschmecken. Linsen als Beilage.

Nach der Zucchini Zeit anstelle der Zucchini mit Porree und / oder Kürbis zubereiten.

\subsection{Brokkoli-Zitronen-Pasta}\label{sec:BrokkoliZitronenPasta}\index{Brokkoli}\index{Pasta!Brokoli}
\subsubsection*{Allgemeines}
\begin{tabular}{ll}
    Personen         &  4   \\
    Zubereitungszeit &  40 Minuten \\
\end{tabular} 
\subsubsection*{Zutaten}
\begin{tabular}{r l}
     1 Stiel & Petersilie                     \\
        15 g & Butter                         \\
        2 EL & Vollkorn-Semmelbrösel (à 10 g) \\
             & Salz                           \\
    2 Zweige & Thymian                        \\
       200 g & Champignons                    \\
           1 & Schalotte                      \\
           1 & Knoblauchzehe                  \\
        2 EL & Olivenöl                       \\
      120 ml & Gemüsebrühe                    \\
       275 g & Schlagsahne                    \\
       250 g & Brokkoli                       \\
             & Pfeffer                        \\
       375 g & Bandnudeln                     \\
         0,5 & Bio-Zitrone
\end{tabular}

\begin{enumerate}
    \item Petersilie waschen, trocken schütteln, Blätter abzupfen und hacken. Butter in einer Pfanne erhitzen und Semmelbrösel darin bei mittlerer Hitze 2–3 Minuten leicht bräunen. Abkühlen lassen, salzen und mit der Petersilie mischen. Thymianzweige waschen und trocken schütteln. Pilze putzen und vierteln. Schalotte und Knoblauch schälen und fein würfeln.
    \item In einer Pfanne Olivenöl erhitzen. Pilze darin 3–4 Minuten bei mittlerer Hitze anbraten. 1 Thymianzweig, Schalotten- und Knoblauchwürfel zugeben und kurz mit dünsten. Mit Gemüsebrühe ablöschen. 250 g Sahne zugießen und alles 5–6 Minuten bei kleiner Hitze köcheln lassen. Derweil Brokkoli waschen, putzen und in kleine Röschen teilen. Sahnesauce mit Salz und Pfeffer würzen und Thymianzweig entfernen.
    \item Zitrone heiß abspülen, trocken reiben, Schale mit einem Sparschäler abschälen und in Streifen schneiden. Restliche Sahne steif schlagen. Vom übrigen Thymianzweig Blättchen abzupfen.
    \item Pasta in der Zwischenzeit in kochendem Salzwasser bissfest garen, dabei für die letzten 3–4 Minuten den Brokkoli dazugeben und mit garen.
    \item Nudeln abgießen, abtropfen lassen und unter die Sauce mischen. Pasta in tiefen Tellern anrichten, mit Semmelbröseln und Zitronenschale bestreuen, je 1 kleinen Klecks Sahne auf die Brokkoli-Zitronen-Pasta setzen und Thymianblättchen darüber streuen.
\end{enumerate}

\subsection{Rote Bete-Eintopf mit Blättern}\label{sec:RoteBete:blaetter}\index{Rote Bete}\index{Rote Bete!Eintopf}
\subsubsection*{Allgemeines}
\begin{tabular}{ll}
    Personen         &  4   \\
    Zubereitungszeit &  40 Minuten \\
\end{tabular} 
\subsubsection*{Zutaten}
\begin{tabular}{r l}
    400 g & rote Beete       \\
    4 & Katoffeln        \\
    3 & Möhren           \\
    2 EL & Olivenöl         \\
    1 l & Gemüsebrühe      \\
    1 Bund & Dill             \\
    3 EL & saure Sahne      \\
    1 & Zitrone          \\
    & Salz und Pfeffer
\end{tabular}

Rote Bete gut bürsten, ungeschält in kleine, 1/2cm dicke Würfel schneiden. Kartoffeln und Möhren schälen und würfeln. Zwiebel schälen und fein würfeln. Dill fein hacken. Rote-Bete-Blätter säubern und in feine Streifen schneiden.

Etwas Öl in einem Topf erhitzen und die Zwiebelwürfel darin glasig schwitzen. Gemüsewürfel hinzugeben und kurz mit anschwitzen. Mit Gemüsebrühe ablöschen und für 15 Minuten köcheln lassen.

Rote-Bete-Blätter hinzugeben und weitere 5 Minuten köcheln lassen. Wenn die Gemüsewürfel weich sind, Dill und saure Sahne (oder Crème fraîche) hinzufügen.

Mit Salz, Pfeffer und Zitronensaft abschmecken.

\section{Fisch}

\subsection{Zitronige Bandnudeln mit Lauch und Lachs}\label{sec:Lachs:LauchBandnudeln}\index{Lachs}\index{Fisch:Lachs}
\subsubsection*{Allgemeines}
\begin{tabular}{ll}
    Personen         &  4   \\
    Zubereitungszeit & 40 Minuten  \\
\end{tabular} 
\subsubsection*{Zutaten}
\begin{tabular}{r l}
        500 g & Lachsfilet                     \\
    2 Stangen & Lauch                          \\
            2 & Schalotten                     \\
            1 & Knoblauchzehe                  \\
         5 EL & Öl                             \\
      3 -4 TL & Kapern                         \\
            1 & Biozitrone                     \\
            5 & Stiele       glatte Petersilie \\
         60 g & Parmesan                       \\
        400 g & Bandnudeln                     \\
              & 
\end{tabular}

Den Backofen auf 160 Grad (Umluft) vorheizen. Den Lachs mit 2 Esslöffeln Olivenöl bepinseln.
Mit Salz und Pfeffer würzen. Die Hälfte des Zitronenabriebs darüber streuen. Auf ein Backblech
legen und im vorgeheizten Backofen etwa 15 bis 20 Minuten garen. 

Die Bandnudeln in reichlich kochendem Salzwasser biss fest garen und abgießen. Etwas Kochwasser auffangen.

Lauch putzen, längs halbieren, waschen und in dünne Scheiben schneiden. Schalotten und Knoblauch schälen und fein würfeln.
Lauch und Schalotten in einer Pfanne im restlichen Olivenöl an schwitzen Knoblauch und Kapern zugeben und kurz durchschwenken. Bandnudeln dazugeben. Mit Zitronenabrieb, -saft und 2 Esslöffeln Nudelwasser vermengen. Mit Salz und Pfeffer würzen.

Lachs aus dem Ofen nehmen und vorsichtig mit einer Gabel zerpflücken. Petersilie waschen, trocken schütteln, die Blätter von den Stielen zupfen und hacken. Zusammen mit dem Lachs unter die Nudeln heben. Parmesan reiben, darüberstreuen und servieren.



\subsection{Seelachs mit Frühlingszwiebeln, Tomaten und Oliven}\label{sec:SeeLachs:TomatenOliven}\index{SeeLachs}\index{Fisch:SeeLachs}
\subsubsection*{Allgemeines}
\begin{tabular}{ll}
    Personen         &  4   \\
    Zubereitungszeit & 40 Minuten  \\
\end{tabular} 
\subsubsection*{Zutaten}
\begin{tabular}{r l}
        800 g & Seelachs                              \\
        400 g & Cocktailtomaten                       \\
       1 Bund & Frühlingszwiebeln                     \\
         5 EL & Olivenöl                              \\
           12 & getrocknete Tomaten in Öl (ca. 150 g) \\
         50 g & schwarze Oliven ohne Kern             \\
          2-3 & Kapern                                \\
    0,5 -1 TL & gemahlene Fenchelsaat                 \\
       0,5 TL & gemahlener Koriander                  \\
       0,5 TL & brauner Zucker                        \\
              & Salz und Pfeffer
\end{tabular}

Die Tomaten waschen und halbieren. Die Frühlingszwiebeln waschen, putzen und in grobe Stücke schneiden. Den Backofen auf 180 Grad (Ober-/Unterhitze) vorheizen.

Ein Backblech mit 2 Esslöffeln Olivenöl einfetten. Den Fisch, die Tomaten und Frühlingszwiebeln darauf verteilen. Die getrockneten Tomaten etwas abtropfen lassen, halbieren und mit den Oliven und Kapern auf dem Blech verteilen. Fenchelsaat, Koriander, Salz und Pfeffer auf den Fisch streuen. Alles mit dem restlichen Olivenöl beträufeln. Den braunen Zucker auf den Schnittflächen der Tomaten verteilen.

Im vorgeheizten Backofen etwa 20 bis 25 Minuten garen. Dazu schmeckt Baguette oder Ciabatta (siehe \vref{sec:brot:Ciabatta:LM} ).

\section{Fleisch}

\subsection{Rinderrouladen}\index{Rindfleisch}\index{Rouladen}\index{Rind!Rouladenbraten}
\subsubsection*{Allgemeines}
\begin{tabular}{lrl}
	Personen         &  4 &  \\
	Zubereitungszeit & 45 & Minuten \\
	Gesamtzeit       &  3 & Stunden \\
	%	Entnommen aus  &    &
\end{tabular} 

\subsubsection*{Zutaten}
\begin{tabular}{rll}
	4 &  & dünn geschnittene Rouladen   \\
	  &  & klein gewürfelte Zwiebeln    \\
	  &  & Gewürzgurken                 \\
	  &  & mittelscharfer Senf          \\
	  &  & Tomatenmark                  \\
	  &  & Salz, Pfeffer, Paprikapulver \\
	  &  & Butterschmalz                \\
	  &  & Rotwein                      \\
	  &  & Gemüsebrühe					\\
	  &  & Sahne \\
	  & & Speisestärke
\end{tabular} 
\subsubsection*{Zubereitung}
\begin{enumerate}
	\item Rouladen waschen und abtrocknen. Von beiden Seiten würzen.
	\item Die Zwiebeln und Gewürzgurken fein würfeln. 
	\item Rouladen auf einer Seite mit Senf (gestrichenen Esslöffel pro Roulade) bestreichen, Zwiebeln und Gurken zugeben und einrollen. Mit Rouladennadeln befestigen.   
	\item in Butterschmalz von allen Seiten kräftig anbraten, Tomatenmark hinzugeben, kurz karamelisieren lassen und mit Rotwein ablöschen.
	\item  Gemüsebrühe hinzugeben (auf ca. $\frac{2}{3}$ Höhe der Rouladen)  und 2 Stunden schmoren lassen. Immer wieder wenden und Flüssigkeit ergänzen.
	\item  Mit Sahne und Speisestärke binden.
\end{enumerate}

\subsection{Florentiner Schweinebraten}\index{Florentiner Schweinebraten}\index{Schweinebraten!Florentiner}\index{Schweinebraten}
\subsubsection*{Allgemeines}
\begin{tabular}{lrl}
	Personen         &   4 &  \\
	Zubereitungszeit &  45 & Minuten \\
	Gesamtzeit       & 105 & Minuten \\
	%	Entnommen aus  &     &
\end{tabular} 

\subsubsection*{Zutaten}
\begin{tabular}{rll}
	          1 & kg       & Kotelettbraten                   \\
	          200 & g       & Champignons                     \\
	            5 & Zweige  & Rosmarin                        \\
	            5 & Zweige  & Thymian                         \\
	$ \frac{1}{2}$ & Bund    & Oregano                        \\
	          1 & Blatt & Lorbeer               \\
	            3 &       & Nelken\\
	              &         & Salz                            \\
	              &         & schwarzer Pfeffer               \\
	            2 & EL      & Olivenöl                          \\
	            1 & TL      & scharfer Senf (z.B. Dijon-Senf) \\
	          300 & ml      & trockener Weißwein              \\
	            1 & TL      & Kartoffelstärke                 \\
	          100 & g       & Sahne                           \\
	            1 & EL      & Weinbrand
\end{tabular} 
\subsubsection*{Zubereitung}
\begin{enumerate}
	\item Den Backofen auf 180 Grad vorheizen. Das Fleisch von Haut und Fettschicht befreien und rundherum mit Salz und reichlich schwarzem Pfeffer aus der Mühle würzen. Den Knoblauch schälen, fein hacken und mit dem Olivenöl verrühren. Das Fleisch damit einreiben. Die Kräuterzweige waschen und trocken schütteln. Die Zweige um die Schweinelende legen und mit dem Küchengarn festbinden. 
	\item Den Braten in einen Bräter legen, 1/8 l Wasser angießen, das Lorbeerblatt und die Nelken dazugeben. Im heißen Ofen (Mitte, Umluft 160 Grad) ca. 1 Std. braten, dabei zweimal wenden. Am Ende soll die Flüssigkeit vollständig verdampft sein. 
	\item Den Braten herausheben und zugedeckt 10 Min. ruhen lassen. Küchengarn und Kräuter vom Fleisch entfernen. Den Braten in Scheiben schneiden und mit dem Kartoffelsalat servieren. 
\end{enumerate}

\subsection{Kalbsfilet à la Zürcher Geschnetzel}\index{Zürcher Geschnetzeltes}\index{Kalb!Zürcher Geschnetzeltes}\index{Kalbsfilet}
\subsubsection*{Allgemeines}
\begin{tabular}{lrl}
	Personen         &  4 &  \\
	Zubereitungszeit & 30 & Minuten \\
	Gesamtzeit       & 30 & Stunden \\
	%	Entnommen aus  &    &
\end{tabular} 

\subsubsection*{Zutaten}
\begin{tabular}{rll}
	600    & g       & Kalbsfilet                      \\
	200    & g       & Champignons                     \\
	2      &         & kleine Schalotten               \\
	2-3    & Zweige  & Thymian                         \\
	einige &         & Estragonblätter                 \\
	4-6    & Stängel & glatte Petersilie               \\
	1      & EL      & Butterschmalz                   \\
	&         & Salz                            \\
	&         & schwarzer Pfeffer               \\
	1      & EL      & Butter                          \\
	1      & TL      & scharfer Senf (z.B. Dijon-Senf) \\
	300    & ml      & trockener Weißwein              \\
	1      & TL      & Kartoffelstärke                 \\
	100    & g       & Sahne                           \\
	1      & EL      & Weinbrand                       \\
\end{tabular} 
\subsubsection*{Zubereitung}
\begin{enumerate}
	\item Am Filet eventuell verbliebene Sehnen mit einem scharfen Messer knapp abschneiden. Das Filet in etwa 1 cm dicke Scheiben schneiden und beiseitestellen.
	\item Die Champignons mit einer Pilzbürste oder Küchenpapier sauber abreiben. Die Stielenden abschneiden und die Champignons halbieren oder vierteln.
	\item Die Schalotten schälen und in feine Würfel schneiden. Thymian, Estragon und Petersilie waschen, trocken schütteln und abzupfen. Thymian und Estragon fein, die Petersilie grob schneiden.
	\item Das Butterschmalz in einer großen Pfanne (28 cm Ø) erhitzen. Die Filetscheiben hineinlegen, mit Salz und Pfeffer würzen und von jeder Seite 1 Min. scharf anbraten. Herausheben und beiseitestellen.
	\item Die Butter in der Pfanne erhitzen und die Schalotten darin bei mittlerer Hitze 2 Min. glasig dünsten. Die Champignons einstreuen und 2 Min. mitbraten. Senf einrühren, Weißwein zugießen und 1 Min. köcheln lassen.
	\item Die Kartoffelstärke mit etwas kaltem Wasser verquirlen, in die kochende Sauce rühren und 2 Min. köcheln lassen.
	\item Die Sauce mit Weinbrand, Salz und Pfeffer abschmecken. Die Filets hineingeben, kurz aufkochen lassen und vom Herd nehmen. Mit den Kräutern bestreuen und noch kurz ziehen lassen. Auf vorgewärmten Tellern servieren.
\end{enumerate}


\subsection{Ofenchili mit Chorizo}\index{Chilli}\index{Ofenchilli}
\subsubsection*{Allgemeines}
\begin{tabular}{lrl}
	Personen         &  8 &  \\
	Zubereitungszeit & 45 & Minuten     \\
	Gesamtzeit       & 45 & Minuten     \\
	Entnommen aus    &    & \cite[vgl.][]{Oliver}
\end{tabular} 

\subsubsection*{Zutaten}
\begin{tabular}{rll}
	   3 &               & Chilischoten                                 \\
	     &               & Möhren                                       \\
	   4 &               & Zwiebeln                                     \\
	   3 &               & Knoblauchzehen                               \\
	 400 & gL            & Chorizo (ersatzweise Metendchen)             \\
	1,25 &               & Gulasch                                      \\
	 1,5 & EL            & Senf                                         \\
	     &               & Salz                                         \\
	     &               & schwarzer Pfeffer                            \\
	   2 & EL            & Tomatenmark                                  \\
	   2 & EL            & Olivenöl                                     \\
	   2 & Dosen         & Tomaten                                      \\
	 1-2 & Dosen         & Kidneybohnen (mal mit eingelegten versuchen) \\
	   2 & Dosen         & Tomaten                                      \\
	   1 & Dose (425 ml) & Mais
\end{tabular} 
\subsubsection*{Zubereitung}
\begin{enumerate}
	\item Chilis längst aufschneiden, entkernen, waschen und in dünne Ringe schneiden. Möhren schälen, waschen und fein würfeln. Zwiebeln und Knoblauch schälen. Zwiebeln würfeln, Knoblauch sehr fein hacken. Von der Chorizo die Haut abziehen, Wurst in Scheiben schneiden. Gulasch gut trocken tupfen.
	\item 	Öl in einem großen Bräter erhitzen. Wurst darin anbraten und herausnehmen. Gulasch in 2–3 Portionen rundum braun braten. Mit Salz und Pfeffer würzen. Möhren, Zwiebeln und Knoblauch zum Schluss kurz mitbraten.
	\item 	Ofen vorheizen (Umluft: 150$^circ$ C). Tomatenmark einrühren und kurz anschwitzen. Alles mit Mehl bestäuben und kurz anschwitzen. Wein, Fond und Tomaten samt Saft zugießen, aufkochen. Chorizo, Chili und alles angebratene Gulasch zugeben. Zugedeckt im heißen Backofen 2 bis 2,5 Stunden schmoren, bis das Fleisch zart und weich ist. Ab und zu umrühren.
	\item 	Kidneybohnen und Mais in ein Sieb geben, abspülen und abtropfen lassen. Ca. 30 Minuten vor Ende der Garzeit zum Chili geben. Dazu passt Baguette. 
\end{enumerate}

\subsection{Gulaschsuppe} \label{sec:Gulaschsuppe}\index{Gulaschsuppe}\index{Suppe!Gulasch}\index{Rindfleisch}
\subsubsection*{Allgemeines}
\begin{tabular}{lrl}
	Personen         &   4 &  \\
	Zubereitungszeit &  45 & Minuten \\
	Gesamtzeit       & ca. 2,5 & Stunden \\
	Entnommen aus    &     & ??
\end{tabular} 

\subsubsection*{Zutaten}
\begin{tabular}{rll}
	0,5 & kg       & Rindfleisch       \\
	200 & g        & Möhren            \\
	200 & g        & Lauch             \\
	250 & g        & Champions         \\
	  2 & m.-große & Zwiebel           \\
	  1 &          & Lorbeerblätter    \\
	  2 & EL       & Tomatenmark       \\
	  3 &          & bunte Paprika     \\
	    &          & schwarzer Pfeffer \\
	100 & ml       & Rotwein           \\
	200 & ml       & Tomatensaft       \\
	300 & ml       & Gemüsebrühe       \\
	    &          & schwarzer Pfeffer \\
	    &          & Rosenpaprika \\
	    &          & Salz \\
	    &          & Paprika\\
	   &        & Petersilie
\end{tabular} 

\subsubsection*{Zubereitung}
\begin{enumerate}
	\item Gulasch klein schneiden (kleiner als Standard). Porree putzen, Zwiebeln schälen,  Möhren würfeln und Pilze kleinschneiden
	\item Fleisch nach und nach anbraten. Gemüse, Lorbeerblatt hinzugeben. Tomatenmark unterrühren. Mit Salz, Pfeffer, Paprika und Rosenpaprika würzen
	\item Mit Mehl bestäuben, anschwitzen und dann mit Brühe, Wein und Tomatensaft ablöschen und aufkochen lassen.
	\item Mit Bräter in den Backofen. 180 Grad 2 Stunden.
	\item Paprika putzen und in Streifen schneiden, Kartoffeln schälen und in Würfel schneiden. Nach ca 30 Minuten hinzufügen.
	\item Ab und zu umruhren und ggf. Flüssigkeit ergänzen.
	\item Abschmecken und mit Petersilie garnieren.
\end{enumerate}



